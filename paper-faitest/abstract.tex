\begin{abstract}
Modern web applications are increasingly data-driven and assist users
with their daily tasks, ranging from browsing social networks to
issuing bank transactions. Users of these applications always expect a
highly personalized experience. This expectation inevitably steers
the collection and processing of data to profile individuals and infer
their preferences. Despite the phenomenal evolution in techniques
and tools to process personal data and infer users'
preferences, developers notoriously miss tools to increase transparency
of this data. Also, developers miss support
to evaluate if policies are enforced correctly across collected data
and if algorithms are creating objectionable biases within user
populations. In this paper we theoretically formalize the
notion of discriminatory treatment of users in modern data-driven web
applications. Then, we design, implement, and evaluate \sysname, a tool
to increase developers’ visibility into the implications of various
data-use policies by reporting discriminatory treatment of users
-- or, {\em privacy bugs}.
Our results show that \sysname can uncover {\em privacy bugs} in the
pricing policy of the ``Staples Inc.'' online store.
\end{abstract}
