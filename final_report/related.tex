\section{Related Work}
\label{sect:related}
Fairness in the use of data by web applications is drawing increasing
attention in our days, and yet there is no established methodology for
measuring fairness of online algorithms producing recommendations, nor is
there an established methodology for shading some light and helping
uncover potential online discrimination cases. The authors of~\cite{Fairness}
study fairness of classification and their purpose is to prevent
discriminating individuals based on certain characteristics, such as
membership in a group, and at the same time maintain the effectiveness of
the classifier. The limitation of their approach is that it requires
programmes to be able to define similarity metrics, and it is unclear
whether programmers are able to define such metrics or not. Also,
cite{DisparateImpact}
unequal treatment of different populations
by a classification methodology is studied for its legal concept of
disparate impact. This work formalizes what it means for a dataset to
have the potential for disparate impact if used by a classifier, and
suggests methods for detecting and removing potential disparate impact.
However, it is not showing the effectiveness of the proposed procedures
in real applications, and also does not extend to general discrimination
hazards.
