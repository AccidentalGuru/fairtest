\section{Motivation}
\label{sect:motivation}
%\begin{itemize}
%  \item Introduce the notion of Fairness in a data driven world. Refer again to the case mentioned
%    in paragraph two (Staples case in detail). Then give specific examples with numbers and tables
%    to simulate discrimination an underline lack (or not) of fairness.
%  \item Analyze the tables and introduce the notion of statistical parity.
%  \item Foreshadow the notion of business necessity.
%  \item Introduce an example with business necessity (requirement).
%  \item Analyze the example based only on statistical parity.
%  \item Contradict and introduce business necessity.
%  \item In addition to statistical parity, analyze also considering
%    business necessity.
%  \item Foreshadow natural inclination/utility function.
%  \item Adreess that we limit our scope without adressing system utility; we leave it for future work.
%\end{itemize}
%

\subsection{Statistical Parity}
For sets of users S and T and some output O, statistical parity asks that:
\begin{equation}
|P\{O | x \in S\} - P\{O | x \in T\}| \le \varepsilon
\label{eq:StatisticalParity}
\end{equation}

Use ~\ref{tab:DiscrimniationExample} and ~\ref{tab:NondiscriminationExample} to motivate.

\begin{table}[h]
{\scriptsize
  \renewcommand{\arraystretch}{1.5}
  \begin{tabular}{ c | c | c  c | c }
    & & \multicolumn{2}{|c|}{\underline{Price}} &  Statistical Parity\\
    Population & \#Members & Low & High & (for high price) \\
    \hline
    A & 30 &  15 & 15 & $0.028 = | \frac{15}{30} - \frac{37}{70}|$ \\
    B & 30 &  14 & 16 & $0.019$ \\
    C & 40 &  19 & 21 & $0.008$ \\
    \hline
    Total & \#100 & 48 & 52 & - \\
  \end{tabular}
  \caption{{\bf Non-discriminatory behavior.} Users of three populations receive approximately
  the same proportions of low versus high prices. Therefore, the probability that a user
  will receive a high price is independent of the population to which he or she belongs,
  and condition~\ref{eq:StatisticalParity} for statistical parity yields a low delta.}
  \label{tab:NondiscriminationExample}
} \end{table}

\begin{table}[h]
{\scriptsize
  \renewcommand{\arraystretch}{1.5}
  \begin{tabular}{ c | c | c  c | c }
    & & \multicolumn{2}{|c|}{\underline{Price}} &  Statistical Parity\\
    Population & \#Members & Low & High & (for high price) \\
    \hline
    A & 30 &  10 & 20 & $0.198 = | \frac{20}{30} - \frac{33}{70}|$ \\
    B & 30 &  16 & 14 & $0.090$ \\
    C & 40 &  21 & 19 & $0.091$ \\
    \hline
    Total & \#100 & 47 & 53 & - \\
  \end{tabular}
  \caption{{\bf Discriminatory behavior against population(s).} Users of population A receive twice
  as many high prices as low, while users of populations B and C receive approximately the same
  ammount of high and low prices. Therefore, the probability that a user will receive a high price
  depends on the population to which he or she belongs, and condition ~\ref{eq:StatisticalParity}
  for statistical parity yields a higher delta compared to the previous example. }
  \label{tab:DiscrimniationExample}
} \end{table}

\subsection{Relaxing Statistical Parity}
There may be situations in which satisfying statistical parity is not reasonable. Indeed, disparate
impact is in itself not necessarily illegal, if the discriminatory practice can be justified
through some notion of business necessity. Although the notion of business necessity was
originally considered for discrimination in hiring, it can reasonably be applied to other situations
of interest such as ad targeting, mortgage or (life) insurance

In many examples where business necessity is brought up, business utility is defined very
simply by the presence of some particular ability or attribute. For instance, a truck
driving company would argue that possession of a truck drivers license is a business necessity,
which would justify that their hiring process is discriminatory against women.
This leads to a simple binary definition of business necessity. Let R denote the set of
users that satisfy the business requirements. Then, a relaxed form of statistical parity would
require that

Essentially, we let the users be discriminated into R and Rc (this is business necessity).
However, we want to make sure that conditioned on being in R (or not), there is no additional
discrimination between sets S and T .
This definition could easily be extended to deal with non-binary categories of utility,
where business-necessity implies splitting the user base into multiple sets



For a sets of users S and T and some output O, relaxed statistical parity asks that:
\begin{equation}
|P\{O | x \in S \cap R\} - P\{O | x \in T \cap R\}| \le \varepsilon
\label{eq:RelaxedStatisticalParityA}
\end{equation}
and
\begin{equation}
|P\{O | x \in S \cap R'\} - P\{O | x \in T \cap R'\}| \le \varepsilon
\label{eq:RelaxedStatisticalParityB}
\end{equation}



Use ~\ref{tab:BusinessNessecity} to motivate.
\begin{table*}[t]
{ \small
  \center
  \renewcommand{\arraystretch}{1.5}
  \begin{tabular}{ c | c c c | c c c | c c c}
    Credit
    & \multicolumn{3}{|c|}{\underline{Loan Type (Population A)}}
    & \multicolumn{3}{|c}{\underline{Loan Type (Population B) }}
    & \multicolumn{3}{|c}{\underline{Loan Type (Population C) }} \\
    history & Payday & Personal & Total & Payday & Personal & Total & Payday & Personal & Total \\
    \hline
    YES & 5  & 15 & 20 & 15 & 40 & 55 & 10 & 45 & 55 \\
    NO  & 80 & 0  & 80 & 45 & 0 & 45 & 45 & 0 & 45\\
    \hline
    Total & 85 & 15 & 100 & 60 & 40 & 100 & 55 & 45 & 100\\
  \end{tabular}
  \label{tab:BusinessNessecity}
  \caption{{\bf Discriminatory behavior on presence of bussiness necessity (credit history).}
  At first sight users of population A are proportionaly taking more payday loans
  (payday loans come with higher interest than personal loans) than users of population B.
  Specifically, users of population A receive 25\% more and 20\% more  payday loans than
  users of populations B and C, respectively. However, upon closer examination, one notes
  that only 20\% of A's users have credit history (which is a prerequisite for personal
  loans) against 55\% of B's and C's users. Therefore, bussiness necessity requires that
  before examining statistical parity, users should be discriminated based on whether
  they have credit history or not.}
}
\end{table*}

\begin{table*}[t]
{ \small
  \center
  \renewcommand{\arraystretch}{1.5}
  \begin{tabular}{ c | c c c | c c | c c}
    Credit
    & \multicolumn{3}{|c|}{\underline{Loan Type (Population A)}}
    & \multicolumn{2}{|c}{\underline{Statistical Parity (Population A) }}
    & \multicolumn{2}{|c}{\underline{Relaxed Statistical Parity (Population A) }} \\
    history & Payday & Personal & Total & Payday & Personal & Payday & Personal \\
    \hline
    YES & 5  & 15 & 20 & - & - &  0.022 & 0.022 \\
    NO  & 80 & 0  & 80 & - & - &  0    & 0 \\
    \hline
    Total & 85 & 15 & 100 & 0.275 & 0.70 & - & - \\
  \end{tabular}
  \label{tab:BusinessNessecityA}
  \caption{{\bf Relaxing statistical parity on presence of bussiness necessity (credit history).}
    Without considering business necessity, i.e., credit history,
    condition~\ref{eq:StatisticalParity} for statistical parity yields a higher delta than if we
    consider business necessity, let the users be discriminated on whether they have credit
    history or not, and apply conditions~\ref{eq:RelaxedStatisticalParityA}
    and~\ref{eq:RelaxedStatisticalParityB}.}
} \end{table*}


