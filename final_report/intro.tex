
\section{Introduction}
\begin{itemize}
  \item One general paragraph to set the ground about data-driven world and apps.
  \item One paragraph to introduce the lack of transparency on data-use
    and the lack of a model to describe and quantify fairness. Brief description of the
    Staples case.
  \item One paragraph to introduce our approach for fairness and our
    idea of reporting privacy bugs.
  \item One paragraph to describe the system we built to measure fairness.
  \item One paragraph describing how we evaluate our system.
  \item One paragraph to describe 3 major contributions of our work
\end{itemize}
%\heading{Motivation.}
%Modern data driven web applications offer enormous capabilities for data
%exchange and usage,
%but there is a strong need for transparency and accountability.
%All the more, highly sensitive personal information are being used from web
%applications to enhance user experience and help create efficient advertising
%policies. Unfortunately, the privacy implications of using that information are
%poorly understood and rarely studied due to lack of transparency and
%accountability. Consequently, there is a clear danger of unintentional data
%misuse, such as algorithmic discriminations. An indicative recent
%example is the Staples price discrimination case~\cite{Staples}.
%In this case an arguably reasonable algorithmic decision to optimize online
%prices based on user proximity to competitor brick-and-mortar locations led to
%consistently higher prices for low-income populations, since they tend to
%generally live farther from these stores. This kind of indirect--and likely
%unintended--effects are challenging to identify, and with the use of sensitive
%information collected by web applications, the risk for these dangers is only
%increasing. Thus, we believe that new programming tools are needed to increase
%developers’ visibility into the data they collect and the implications of its
%integration into applications.
%
%\heading{Goals.}
%Our purpose is to build \thetool: A discrimination testing suite uncovering
%differential treatment of various populations based on protected
%attributes, such as religion, sexual orientation, or income. Our idea is to
%measure correlation between outputs of a specific program (such as a price
%recommendation engine) and the values of the protected attributes. Any strong
%correlation between protected attributes and outputs is threated as a privacy
%bug and is reported to the
%developer as potential discrimination. To provide a better perspective to the
%programmer, we will try to develop meaningful heuristics that rank observed
%correlations based on their likelihood to be unintended or unfair
%discriminations versus natural, preference-based effects. We will design
%{\it FairTest} to be easy to use by regular application programmers,
%will integrate
%it into popular testing frameworks, and write a set of sample applications
%indicating meaningful use cases.
%
%\heading{Expected Conclusions.}
%By using \thetool a developer will be able to test an application for
%discrimination cases by uncovering correlations between protected attributes
%and outputs. We expect to be able to identify clear cut discrimination cases,
%as well as previously unknown obscure discrimination cases. \thetool should
%work independently from the backend algorithms used for recomendations,
%and it should feature a
%simple API allowing developers easily test their code for privacy bugs.
%
%%\heading{Research Plan.}
%%The first milestone is to build a demonstrating scenario for the Staple
%%case~\cite{Staples} no later than the project status presentation date, on 3/10.
%%This will include a demo set of users created by tuning yelp's publicly
%%available data set~\cite{Yelp} to resemble Staple's user set, and a list of
%%random competitor addresses. Then, we will build a sample application
%%recommending product prices based on user and competitor locality.
%%The second milestone is to build the necessary infrastructure to identify and
%%rank correlations between user attributes, such as address, sex, and age, with
%%outputs, such as price. The third milestone is the design of \thetool's API
%%for writing tests to rank the aforesaid correlations and the integration of
%%\thetool's API with a popular web framework, such as Django. We intend to write
%%some basic test cases --unittests-- to demonstrate clear cut cases of strong
%%correlations between inputs and outputs. The fourth milestone, is to present our
%%experiences and lessons learned as well as interesting (and previously unknown)
%%unintentional cases of discrimination uncovered by \thetool.
