\documentclass{article}

\usepackage{xspace}
\usepackage{hyperref}
\usepackage[margin=1in]{geometry}
\usepackage{listings}

\newcommand{\thetool}{{\it FairTest}\xspace}
\newcommand{\heading}[1]{\noindent{\bf{#1}}}
\newcommand{\xxx}[1]{\textbf{{XXX} #1}}

\title{\thetool: A Unit Testing Framework for Uncovering Privacy Bugs in
  Internet-of-Things Applications}
\author{V. Atlidakis, D. X. He, E. D. Park}
\date{\today}

\begin{document}
\maketitle

\heading{Motivation.}
The Internet of Things (IoT) offers enormous capabilities for data exchange
enhancing societal progress, but it comes with a strong need for transparency
and accountability with respect to data usage. All the more, highly sensitive
data are transmitted from devices on our bodies, in our homes, and in the
streets. Unfortunately, the privacy implications of collecting and using that
information are poorly understood and rarely studied. Consequently, there is a
clear danger of unintentional algorithmic discriminations. An indicative recent
example is the Staples price discrimination case~\cite{Staples}.
In this case an arguably reasonable algorithmic decision to optimize online
prices based on user proximity to competitor brick-and-mortar locations led to
consistently higher prices for low-income populations, since they tend to
generally live farther from these stores. This kind of indirect--and likely
unintended--effects are challenging to identify, and with the use of sensitive
information collected by IoT applications, the risk for these dangers is only
increasing. Thus, we believe that new programming tools are needed to increase
developers’ visibility into the data they collect and the implications of its
integration into applications.

\heading{Goals.}
Our purpose is to build \thetool: A discrimination testing suite uncovering
differential treatment of various populations based on protected
attributes, such as religion, sexual orientation, or income. Our idea is to
measure correlation between outputs of a specific program (such as a price
recommendation engine) and the values of the protected attributes. Any strong
correlation between protected attributes and outputs is reported to the
developer as potential discrimination. To provide a better perspective to the
programmer, we will try to develop meaningful heuristics that rank observed
correlations based on their likelihood to be unintended or unfair
discriminations versus natural, preference-based effects. We will design
{\it FairTest} to be easy to use by regular app programmers, will integrate it
into popular testing frameworks, and write a set of sample applications
indicating meaningful use cases.

\heading{Prior Experience.}
\xxx{prior work}
Fairness in the usage of data by IoT applications is drawing
increasing attention in our days, and yet there is no established methodology
for measuring fairness of online algorithms producing recommendations, nor is
there an established methodology for shading some light and helping uncover
potential online discrimination cases.
The authors of~\cite{Fairness} study the fairness of classification and their pupose is
to prevent discrimitating of individuals based on certain characteristics, such as
membership in a group, and at the same time maintain the effectiveness (utility)
of the classifier. The limitation of this approach is that it requires programmes
to be able to define similarity metrics and it is unclear to what extend programmers
will be able to define such metrics.
Also, in~\cite{DisparateImpact} unuqual threatment of different subpopulations by a
classification methodology is studied for its legal concept of disparate impact.
This work formalizes what it means for a dataset to have the
potential for disparate impact if used by a classification mechanism,
and suggests procedures for detecting and removing
potential disparate impact in a dataset. However, it is not showing the
effectiveness of the proposed procedures in real applications, and also does
not extend to more general discrimination hazards that we consider here.


\heading{Expected Conclusions.}
By using our framework, a developer will be able to test an application of discrimination. Failing 
the test would inform the developer that output of the algorithm used by the application has 
significant correlation with the private data of users. Passing the test should indicate that the application 
is free of discrimination with reasonable probability. The framework should work independently from 
the algorithm being tested so that it can be used to test various different algorithms, and the api 
should be simple so that developers can easily test their algorithm, whenever they need it.

\heading{Research Plan.}
The first milestone is to build a prototype of our system no later than
the project status presentation date, on 3/10. Our prototype will include a
demo dataset for an application imitating a known discrimination case, such as
the Staple~\cite{Staples} case. We expect to be able to identify strong
correlations between protected attributes and outputs in clear cut cases such
as in~\cite{Staples}. The second milestone is to augment our prototype
with real datasets (hopefully obtained from online web stores) and a set of
simple applications featuring a real price recommendation engine. Our hope is
that we will be able to uncover obscure correlations of protected attributes
and outputs that are impossible to foresee by the programmer. The third
milestone is to finalize a fine grained API for integrating our \thetool
framework in a popule testing framework. The fourth milestone, is to present
our experiences and lessons learned as well as interesting (and previously
unknown) unintendiotal cases of discrimination uncaovered by \thetool.

{
  \scriptsize
  \setlength\itemsep{0pt}
  \footnotesize
  \bibliographystyle{abbrv}
  \bibliography{whitepaper}
}
\end{document}
